\section{Outline of Contents}\label{outline-of-contents}

Part 1 of this International Standard consists of the following:

\begin{enumerate}
\def\labelenumi{\arabic{enumi}.}
\item
  The syntax and semantics of the OMG interface definition language (OMG
  IDL), which is used to describe the interfaces that client objects
  call and object implementations provide. Throughout this specification
  the abbreviation IDL is used, for brevity, as shorthand for OMG IDL.
\item
  The interface to the ORB functions that do not depend on object
  adapters: these operations are the same for all ORBs and object
  implementations.
\item
  The semantics of passing an object by value.
\item
  An IDL abstract interface, which provides the capability to defer the
  determination of whether an object is passed by reference or by value
  until runtime.
\item
  The Dynamic Invocation Interface (DII), the client's side of the
  interface that allows dynamic creation and invocation of request to
  objects.
\item
  The Dynamic Skeleton Interface (DSI), the server's-side interface that
  can deliver requests from an ORB to an object implementation that does
  not have compile-time knowledge of the type of the object it is
  implementing.
\item
  The interface for the Dynamic Any type that allows statically-typed
  programming languages such as C and Java to create or receive values
  of type Any without compile-time knowledge that the typer contained in
  the Any.
\item
  The Interface Repository that manages and provides access to a
  collection of object definitions.
\item
  The Portable Object Adapter that defines a group of IDL interfaces
  that an implementation uses to access ORB functions.
\item
  ORB operations that allow services such as security to be inserted in
  the invocation path.
\item
  Messaging which covers: Quality of Service, Asynchronous Method
  Invocations (to include Time-Independent or ``Persistent'' Requests),
  and the specification of interoperable Routing interfaces to support
  the transport of requests asynchronously from the handling of their
  replies.
\end{enumerate}

\section{Keywords for Requirement Statements}\label{keywords-for-requirement-statements}

The keywords ``must,'' ``must not,'' ``shall,'' ``shall not,''
``should,'' ``should not,'' and ``may'' in this specification are to be
interpreted as described in \cite{rfc2119}.

% \REPLACEME{File 6\_Additional.tex}
% 
% \section{How to read this Specification}
% \REPLACEME{A quick description of how to read this document, stating what it describes, and then what is in each section/clause.
% 
% Example:
% 
% This \textbf{specification} presents a metamodel for describing bird migration, Bird Land And Sea Tracking (BLAST).
% Clauses 1 to 6 provide compliance rules, terms definitions and reference information.
% Clause 7 provides the description of the metamodel for BLAST.
% Clause 8 provides an informative example of how to use BLAST to track flock patterns.
% Clause 9 provides informative examples of how to define individual routes in BLAST, with equivalent XML representations.}
% 
% All clauses of this document are normative unless explicitly marked ``(informative)''. The marking ``(informative)'' of a particular clause applies also to all contained sub-clauses of that clause.
% 
% \REPLACEME{To mark a clause (informative), follow the chapter, section, or subsection title with (informative), as in this Section.}
% 
% \section{Acknowledgments}
% \REPLACEME{REQUIRED: Give formal acknowledgements here, starting with submitters and then working through contributors and supporters.  Delete either or both of the last two sections if there are no entries.}
% 
% The following organizations submitted this specification: 
% 
% \begin{itemize}
% \item Company X
% \item Company Y
% \end{itemize}
% 
% The following organizations contributed to this specification: 
% 
% \begin{itemize}
% \item Company A
% \item Company B
% \end{itemize}
% 
% The following organizations supported this specification: 
% 
% \begin{itemize}
% \item Company 1
% \item Company 2
% \end{itemize}


%%% EXPERIMENTAL BELOW THIS POINT
% Go read sec 13.7 on conditionals *CAREFULLY*: http://mirrors.ibiblio.org/CTAN/info/texbytopic/TeXbyTopic.pdf
% % \newcommand{\Foo}[1]{#1}
% % \newcommand{\Bar}{A}

% % \ifthenelse{\equal{\Foo{A}}{\Bar}}{Same}{Not}

% \lengthETB\contributors

% % \ifthenelse{0=\expandafter\lengthETB\contributors}{Foo}{Bar}
% % \setvaluelist{\contributors}{{A}}
% \lengthETB{\contributors}

% \setvalue{\foo}{\lengthETB{\contributors}}
% \foo

% \ifstrequal{\lengthETB{\contributors}}{0}{Empty}{Not Empty}

% % \expandafter\ifstrequal\expandafter{\expandafter\expandafter\foo}{0}{Empty List}{Not empty list}


% % \ifthenelse{\value{\foo}=0}{Empty}{Not Empty}

% % \ifstrequal\foo0{Empty List}{Not empty list}

% % \expandafter\ifstrequal\expandafter{\foo}{0}{Empty List}{Not Empty List}

% % \setvalue{\foo}{\lengthETB{\contributors}}

% % \foo

% % \ifthenelse{\equal{\lengthETB{\contributors}}{0}}{Foo}{Bar}

% % \ifnum 0 = {\foo}{Foo}\else{Bar}\fi
\vfill\eject

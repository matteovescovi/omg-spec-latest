%%%
%%% All author customizations should go in this file
%%%

%%%---------------------------------
%%% DRAFT watermark and commentary control
%%%
%%% These two lines, if uncommented, will remove the watermark and margin comments
%%% to prepare your document for final formatting
% \SetWatermarkText{}
% \setboolean{hidecomments}{true}

%%%---------------------------------
%%% Customize location of Model-Generated content, if using MDSA tools
%%%
%%% Uncomment this and provide the path to *model-generated* .tex and .svg content
%%% if you are placing somewhere other than the standard locations entered here...
%  \renewcommand{\genfiles}{GeneratedContent/}
%  \graphicspath{{../GeneratedContent/Images/}}


%%%---------------------------------
%%% Bibliography References
%%% See https://www.overleaf.com/learn/latex/Bibliography_management_in_LaTeX
%%% for guidance
%%%
%%% Location of the .bib file to provide your own bibliography entries
%%% If you have additional .bib files you may add them by copying the line and
%%% updating the filename. Note that you must include ".bib" at the end of the
%%% filename.
\addbibresource{specification.bib}

%%% OMG publication guidelines strongly suggest that references be broken out into
%%% normative, and non-normative lists.  By default, cited references will appear
%%% in the NON-normative bibliography. Add bibliographic citation references to 
%%% this list to have them appear in the normative bibliography instead.
% \addtocategory{normative}{MOF25, MOF251, UML251, XMI, XMLSchema}

%%% LaTeX will build your bibliography section from the references you cite in your
%%% document. If you wish to include certain 'default' references without needing to 
%%% explicitly find a place to add a \cite command, you may add them to the following
%%% list instead.
% \nocite{MOF25, MOF251, UML251, XMI, XMLSchema}


%%%---------------------------------
%%% Useful but completely optional packages
%%% Feel free to comment out the following sections if you are not using them.

%%% Source code listings, with useful defaults
%%% https://www.overleaf.com/learn/latex/Code_listing#Using_listings_to_highlight_code
\usepackage{listings}
\lstset{breaklines=true, tabsize=4, captionpos=b, basicstyle=\ttfamily}

%%% Support for tables that span across multiple pages
%%% https://www.overleaf.com/learn/latex/Tables#Multi-page_tables
\usepackage{longtable}

%%%---------------------------------
%%% User added packages
%%%
%%% You may add any additional changes you like below this line
